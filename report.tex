\documentclass[a4paper,11pt,twoside]{article}
%\documentclass[a4paper,11pt,twoside,se]{article}

\usepackage{UmUStudentReport}
\usepackage{verbatim}   % Multi-line comments using \begin{comment}
\usepackage{courier}    % Nicer fonts are used. (not necessary)
\usepackage{pslatex}    % Also nicer fonts. (not necessary)
%\usepackage{lmodern}   % Optional fonts. (not necessary)
%\usepackage{tabularx}
%\usepackage{microtype} % Provides some typographic improvements over default settings
%\usepackage{placeins}  % For aligning images with \FloatBarrier
%\usepackage{booktabs}  % For nice-looking tables
%\usepackage{titlesec}  % More granular control of sections.

% DOCUMENT INFO
% =============
\department{Institution för Datavetenskap}
\coursename{Datavetenskapens byggstenar 7.5 p}
\coursecode{DV160HT15}
\title{OU1 Testing}
\author{Lorenz Gerber  ({\tt{dv15lgr@cs.umu.se}})}
\date{2015-11-17}
%\revisiondate{2015-09-15}
\instructor{Lena Kallin Westin / Johan Eliasson}


% DOCUMENT SETTINGS
% =================
\bibliographystyle{plain}
%\bibliographystyle{ieee}
\pagestyle{fancy}
\raggedbottom
\setcounter{secnumdepth}{2}
\setcounter{tocdepth}{2}
%\graphicspath{{images/}}   %Path for images

% DEFINES
% =======
%\newcommand{\mycommand}{<latex code>}


% DOCUMENT
% ========
\begin{document}
\maketitle

\tableofcontents

\section{Introduction} 
% Define the subject of the report: 
%    "Why was this study performed?"
% Provide background information and relevant studies:
%    "What knowledge already exists about this subject?"
% Outline scientific purpose(s) and/or objective(s): 
%    "What are the specific hypotheses for investigation?"

\section{Material and Methods} 
% List materials used, and how were they used. 
% Provide enough detail for the reader to understand the experiment

\section{Results} 
% Concentrate on general trends and differences and not on trivial details.
% Summarize the data from the experiments without discussing their implications
% Organize data into tables, figures, graphs, etc. 
% * Title all figures and tables; include a legend explaining symbols, abbreviations, or special methods
% * Number figures and tables separately and refer to them in the text

\section{Discussion} 
% Interpret the data; do not restate the results
% Relate results to existing theory and knowledge
% Explain the logic that allows you to accept or reject your original hypotheses
% Include suggestions for improving your techniques or design, 
% or clarify areas of doubt for further research

\addcontentsline{toc}{section}{\refname}
\bibliography{references}

\end{document}
