\documentclass[a4paper,11pt,twoside]{article}
%\documentclass[a4paper,11pt,twoside,se]{article}

\usepackage{UmUStudentReport}
\usepackage{verbatim}   % Multi-line comments using \begin{comment}
\usepackage{courier}    % Nicer fonts are used. (not necessary)
\usepackage{pslatex}    % Also nicer fonts. (not necessary)
%\usepackage{lmodern}   % Optional fonts. (not necessary)
%\usepackage{tabularx}
%\usepackage{microtype} % Provides some typographic improvements over default settings
%\usepackage{placeins}  % For aligning images with \FloatBarrier
%\usepackage{booktabs}  % For nice-looking tables
%\usepackage{titlesec}  % More granular control of sections.

% DOCUMENT INFO
% =============
\department{Institution för Datavetenskap}
\coursename{Datavetenskapens byggstenar 7.5 p}
\coursecode{DV160HT15}
\title{OU1 Testing}
\author{Lorenz Gerber  ({\tt{dv15lgr@cs.umu.se}})}
\date{2015-11-17}
%\revisiondate{2015-09-15}
\instructor{Lena Kallin Westin / Johan Eliasson}


% DOCUMENT SETTINGS
% =================
\bibliographystyle{plain}
%\bibliographystyle{ieee}
\pagestyle{fancy}
\raggedbottom
\setcounter{secnumdepth}{2}
\setcounter{tocdepth}{2}
%\graphicspath{{images/}}   %Path for images

% DEFINES
% =======
%\newcommand{\mycommand}{<latex code>}


% DOCUMENT
% ========
\begin{document}
\maketitle

\tableofcontents
\newpage

\section{Introduction} 
The aim of this laboration is to write unit test code for an
implementation of the data type `queue'. The unit tests shall 
return feedback to the user, indicating `success' or `failure'. The
interface of the datatype may not be changed. All operations given in
\cite[p. 155]{janlert2000} shall be tested.

\section{Material and Methods} 
The datatype `queue' was said to be implemented according to the 
specifications given in \cite[pp.155 -- 172]{janlert2000}. The
`C' implementation is according to `C99' standard. I chose to 
implement the unit tests according to the \textit{axiomatic specification} 
given in \cite[pp.156 + 157]{janlert2000}. The toolchain was \textit{clang}, 
\textit{CMake 3.3.2}, \textit{GDB 7.8}, IDE was \textit{CLion
  1.2.1}. \textit{Valgrind} was used to find memory leaks. Development platform was \textit{OSX 10.10}. 

\section{Results}
\subsection{General}
The source code was stored in a file \texttt{queuetest.c} and submitted
according instructions on the `cambro' web interface. The unit tests were
all implemented as \texttt{void} functions taking no arguments. All function
output went to \texttt{stdout}. Each function outputs a short description of 
the tested operation. Depending on success, `pass' or `fail' is written
out. In case of `fail', the function exits and returns the value
\texttt{1}. All test functions are called from \texttt{main.c}. 
\subsection{Unit Tests - Formal Descriptions}
\begin{enumerate}
  \item Axiom 1, use \texttt{empty} to create an empty list and check
    with \texttt{isEmpty} if it is really emtpy. 
    \item Axiom 2, apply \texttt{enqueue} to an empty list and check
      that \texttt{isEmpty} returns \texttt{FALSE}.
      \item Axiom 3, if a queue q is empty, it follows that 
        consecutive \texttt{enqueue}, \texttt{dequeue} will result in
        the same queue q. 
        \item Axiom 4, if a queue q is not empty, it follows that
          \texttt{dequeue} and \texttt{enqueue} follow commutative
          properties, hence the resulting queue q will look the same
          independent in which sequence \texttt{dequeue} and 
          \texttt{enqueue} are applied.
          \item Axiom 5, if a queue q is empty, sequential
            application of \texttt{enqueue(v, q)} and
            \texttt{front(q)} will return \texttt{v}.
            \item Axiom 6, if a queue q is not empty,
              \texttt{enqueue} will not affect the next \texttt{front}
              operation. 
\end{enumerate}

\section{Discussion}

\subsection{Unit Tests - Interpretation}
\begin{enumerate}
\item Test 1: Fails if \texttt{empty} does not work. If
  \texttt{isEmpty} was hard coded to \texttt{TRUE}, this test could
  still be passed.

\item Test 2: If this test passes, we know that \texttt{isEmpty}
  works, hence we know also that \texttt{enqueue} can add a value to
  the queue. Moreover we can be sure that \texttt{Empty} works
  properly. If the test fails, either \texttt{isEmtpy} was hardcoded,
  \texttt{Empty} did not create a valid, empty queue or
  \texttt{enqueue} did not add a value to the empty queue.

\item Test 3: If test 3 fails, there is something wrong with
  \texttt{dequeue}. If test 3 passes, we know that both
  \texttt{enqueue} and \texttt{dequeue} can add respectively remove a
  value from the queue. But we don't know anything whether they act on
  the correct end of the queue.

\item Test 4: If it fails, we know that either \texttt{enqueue} or
  \texttt{dequeue} act on the wrong end of the queue. However, if both
  act on the wrong end, the test will still pass.

\item Test 5: when it fails, \texttt{front} can not read correctly a
  value from the queue. If it passes, we still don't know if it reads from the
  correct end.

\item Test 6: If it passes, we know that \texttt{front} reads
  from the correct end in relation to \texttt{enqueue} and
  \texttt{dequeue}. Hence also the latter two act on the correct end.
\end{enumerate}

Remark: In the end we know that \texttt{enqueue}, \texttt{dequeue} and
\texttt{front} work correctly in relation to each other. But the
implementation could still be reversed to how it was expected. As
the queue is implemented with a list, this poses no problem and the
datatype will still work as expected. However, when the queue would be
implemented on top of a static datatype such as array, this could mean
trouble. Adding a test 8, could check the absolute correct acting ends
of all operations. This was not implemented in the current unit tests.
 
\subsection{Dynamic Memory Handling}
The given implementation of queues with a `2-cell list'  uses dynamic memory
allocation in C. Memory handling functions are implemented. Therefore,
when comparing the outcome of two independent queues, two sets of data
have to be prepared. Operations such as \texttt{dequeue} will
deallocate dynamic memory hence the data will not be available for the
other queue operation. This was the case for Axiome 4 and 6.

\subsection{Application of Unit Tests to Provided Datatype}
The implemented unit tests where applied to the provided datatype
\texttt{queue} (implemented with a `2-cell list'). Here, the unit tests 
all passed.

\subsection{Provoking Unit Test Fails}
Several modifications on the given datatype were tested to provoke
unit test fails. Modifications were always tested on the whole chain
of unit tests. Each modifcation stated below was tested separate. 

\begin{enumerate}
\item Test 1, \texttt{isEmpty)} was modified to always 
return \texttt{FALSE}.
\item Test 2, \texttt{isEmpty} was modified to always
    return \texttt{TRUE}.
\item Test 3, in \texttt{enqueue} the function call
  to the list function was removed.
\item Test 4, in \texttt{dequeue}, the function call
  was modfied to show stack behaviour, entries were removed on the
  previous to last list position.
\item Test 5, in \texttt{front}, the function call
  was removed and instead the address to a hardcoded \texttt{int} value
  returned. This resulted in a compiler warning as a local stack
  variable was passed on. It worked however to provoke a fail for axiom
  5.
\item Test 6, for both \texttt{enqueue} and \texttt{dequeue} the
  position of action was changed: \texttt{enqueue} added at first list
  position, while \texttt{dequeue} removed the previous to last.
\end{enumerate} 

\subsection{Memory Leaks}
Valgrind was used to detect and mend memory leaks, both in `pass' and
`fail' cases. This was to test the actual unit tests. However, if
there are errors in the memory handler of the datatype, the unit tests
will not detect it. But also here, valgrind will server as tool.

\addcontentsline{toc}{section}{\refname}
\bibliography{references}

\end{document}
